The Internet of Things (IoT) has stimulated the creation of a new era with the promise of ubiquitous connectivity aided by a proliferation of inexpensive IoT devices. The threat presented by the recent rise of botnets and their capability to infect numerous vulnerable IoT devices on the internet mandate a better understanding of their inner workings and investigation of alternative defensive mechanisms usable by \textbf{resource-constrained IoT devices}. In this project, we aim to model and simulate the dynamic behavior of the Mirai botnet as a network of timed automata using the modeling tool UPPAAL to delve deeper into the botnet infection process. Additionally, to determine its feasibility as a defensive measure, we examine the effectiveness of rebooting and network degradation on varying system configurations. The resulting formal analysis provides a solid understanding of the impact of such mechanisms on botnet growth and insight into their efficacy to weigh them against other available countermeasures.


% \torevise{The Internet of Things (IoT) has stimulated the creation of a new era with the promise of ubiquitous connectivity. Recent years have seen a proliferation of cheap IoT devices, many of which employ energy-efficient designs, mandating low computational resources that preclude the use of rigorous security mechanisms. Furthermore, many deployed IoT devices continue to use default vendor passwords, and many others tend to have weak credentials. Adversaries in IoT exploit these security vulnerabilities to create their own army of connected IoT devices, known as botnets, only to be used later to perform various malicious attacks on the network. In this project, we aim to model and simulate the dynamic behavior of a botnet as a networked timed automata using the modeling tool UPPAAL to gain insight into the botnet infection process in diverse system configurations. Additionally, we aim to analyze the impact of the infection process on the modeled network by implementing existing security mechanisms and assess the feasibility of such measures.}

% The threat presented by the recent rise of botnets and their capability to infect numerous vulnerable IoT devices on the internet mandates a better understanding of their inner workings and investigation of alternative defensive mechanisms usable by resource-constrained IoT devices. 

% The knowledge gained through this study allowed us to reinforce our understanding of the botnet infection process as well as identify areas of limited understanding such as suggested frequency and duration of device rebooting. Additionally, they will gain insight into the effectiveness and viability of rebooting and weigh it against other available defensive measures to provide suitable recommendations.

