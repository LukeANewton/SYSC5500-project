The challenges encountered during the project come from two sources: technological limitations, and a lack of available studies on IoT device execution and rebooting times.~\par

At the time of writing, UPPAAL is available only as a 32-bit program on devices running Windows, which were the only device types available to us. This, coupled with UPPAAL's single-threaded simulations, led to a software bottleneck that ultimately limited what was feasible to simulate on our devices. UPPAAL also stores numbers as 16-bit signed integers, which became problematic as we were simulating many small interactions over a long time. It was necessary to include additional variables for the number of integer overflows, which allowed us to model a higher maximum value, but made recording values and specifying parameters more complicated.
\par
%The two factors that affect device execution timings and network latency, are both difficult to estimate. The default passwords used by Mirai have been mapped to the devices and vendors~\cite{Antonakakis2017_USENIX_Mirai_First_Study}, so we considered using this with knowledge of those vendors' devices to reason about execution times. In the end, we determined that this is likely negligible compared to latency as the propagation process is computationally simple and latency can vary wildly for different networks, so we selected a conservative time of 100 ms for most simulations.

The overall timing of the propagation process is determined by device execution times and network latency, which are both difficult to estimate. The default passwords used by Mirai have been mapped to the devices and vendors~\cite{Antonakakis2017_USENIX_Mirai_First_Study}, so we considered using this with knowledge of those vendors' devices to reason about execution times. In the end, we determined that this is likely negligible compared to latency as the propagation process is computationally simple. Latency can vary wildly for different networks, so we selected a conservative time of 100 ms for most simulations.

\par
Reboot frequency and the duration of a device's reboot process were also difficult to determine. We explored works on rebooting times, user behaviors, and manufacturer recommendations for reboot frequency, but were mostly unsuccessful. A few IoT device manuals~\cite{comtrend_2012} suggest that their reboot process takes about 60 seconds, so we decided to use this time in our model. As we were unable to find any concrete information on the recommended reboot frequency, we adjusted our research focus to see how varying this affects the botnet size.
\par
We should note that UPPAAL is a research tool and is under active development, so many of the technical challenges may be solved in future versions of the tool. Access to faster computation resources and alternative operating systems would allow for a wider range of scenarios to be tested on our model. The model is designed to allow timings to be altered easily, so if papers are published in the future on IoT rebooting behaviors, the model can be updated accordingly to obtain potentially more accurate results.