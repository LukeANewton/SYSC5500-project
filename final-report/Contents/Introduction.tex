% ====================================== Introduction to IoT =========================================
The Internet of Things (IoT) refers to the interconnected network of a vast number of internet-enabled physical objects around the world. The increased availability of low cost, low power sensors, and enhanced communication methods has caused an explosion in the number of IoT devices in recent years~\cite{atzori2010_IoT_Survey}. With this increased communication, data collection, and analysis, IoT has the potential to improve many facets of society, including, but not limited to, healthcare, infrastructure, supply chains, and the general home and office environments. Unfortunately, the relatively low computational resources available for typical IoT devices, coupled with a rushed productive mentality, have made the IoT infrastructure vulnerable to a number of security threats~\cite{alaba2017_IoT_Security_Survey}.

% ===================================== Introduction to Botnets ======================================
The sudden proliferation of IoT devices has also garnered the attention of vulnerable device herders known as botmasters. These attackers use the inherent and user-induced vulnerability present in exposed end-points in the IoT infrastructure to take control of these devices to later use them for nefarious purposes. One prime example is the Mirai botnet attack that affected millions of poorly configured IoT devices to stage a massive distributed denial-of-service (DDoS) attack in 2016~\cite{bertino2017_Botnets_IoT}. The last few years have seen a surge in the frequency and sophistication of botnet attacks, resulting in varying levels of impact on the individual devices, services, and the network alike~\cite{kolias2017_Mirai_DDoS, kambourakis2017mirai}.

% ==================== Why understanding the Botnet infection process is crucial =====================
The defense against botnets involves the meticulous inspection of the intricate interaction among the entities inside the botnet. The Mirai botnet infrastructure, the way devices are infected, controlled, and used to infect or perform other malicious actions, have been studied in great detail by the security community~\cite{kolias2017_Mirai_DDoS, Antonakakis2017_USENIX_Mirai_First_Study}. However, the analyses performed on the impact of such botnets, including Mirai, are mostly evidentiary, meaning we do not have sufficient information to predict the aftermath in the event of a botnet reaching an unprecedented size. Furthermore, many gray areas remain in determining all the distinct types of devices that were affected, limiting the flexibility and extensibility of such analyses.

% ===================================== Why use formal modeling ======================================
Formal models have proven to be quite useful in assuring the security of a system. The development of a formal botnet model can considerably improve the understanding of botnet capabilities and how best to deal with them. The advantages of using a formal specification language are manifold: the correctness, as well as various properties of the model, can be verified using the full range of available formal methods and discover areas of ambiguity in the literature. Furthermore, we can model both individual device behaviors as well as see how different distributions of clusters of such devices affect the infection process, and in turn, the impact of botnets. Most importantly, the high decidability of specific modeling formalisms allows achieving highly accurate results and predictive capabilities.

% ================================= Bulleted List: core contribution =================================
The main contributions of this project are as follows.
\begin{itemize}
    \item Modeling the individual entities in the Mirai botnet infrastructure and the behavior of different device clusters in an IoT network using timed automata.
    \item Simulation of the real-time infection and propagation of botnets in a network of heterogeneous devices with varying distributions to estimate the impact.
    \item Formal analysis of the effectiveness and feasibility of rebooting and network degradation as proposed measures to curb the spread of botnets under varying scenarios. 
\end{itemize}
\par

% ==================================== Organization of the paper =====================================
The rest of this paper is structured as follows. Section~\ref{sec:background} discusses the modeling terminologies and concepts used for the remainder of the paper. Section~\ref{sec:botnets} examines botnets and their malicious capabilities, and section~\ref{sec:modeling} outlines the botnet infection process along with our developed model. Section~\ref{sec:experimental_results} details simulation parameters and the results from specific simulation runs, while Section~\ref{sec:discussion} discusses the implications of those results. Section~\ref{sec:challenges} discusses the challenges faced during the project. Section~\ref{sec:literature_review} provides an overview of pertinent botnet studies to date. Finally, Section~\ref{sec:conclusion} concludes our work with a brief discussion of possible future extensions.



% ======================================= Proposal Info ==============================================
%  \mynote{1}{From the problem statement of the project proposal:}
% In this project, we aim to model a Mirai-like botnet infection process using timed automata and observe the impact of the infection on the simulated network in varying system configurations. The timed automata mathematical model allows us to formally model and analyze the real-time behavior of botnets and its individual components, and observe the behavioral changes in the botnet due to behavioral changes in individual entities within the network~\cite{alur1994_Theory_Timed_Automata}. The high decidability of timed automata allows us to exhaustively check the model for reachability properties such as whether in each simulation scenario, the botnet is capable of infecting all the devices within the entire network, and if not, why. Another consideration would be identifying the percentage of devices that can consistently be infected within a certain time in each system configuration, and how implementing existing security mechanisms impact those figures. 

%  \mynote{1}{From the work plant of the project proposal:}
%  For modeling purposes, we will be using the integrated tool environment UPPAAL that allows for the modeling, verification, and validation of real-time systems modeled as networked timed automata~\cite{behrmann2004_Tutorial_UPPAAL}. UPPAAL is freely available only for research purposes\footnote{http://www.uppaal.org/} within the academia.