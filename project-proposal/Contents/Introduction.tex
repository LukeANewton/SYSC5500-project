 
The Internet of Things (IoT) refers to the interconnected network of the vast number of internet-enabled physical objects around the world. In recent years, the increased availability of low cost, low power sensors, and enhanced communication methods has caused an explosion in the number of IoT devices~\cite{atzori2010_IoT_Survey}. With this increased communication, data collection, and analysis, IoT has the potential to improve many facets of society, including, but not limited to, healthcare, infrastructure, supply chains, and the general home and office environments. Unfortunately, due to the relatively low computational resources available for typical IoT devices and the rushed production of many such devices, the IoT infrastructure has become vulnerable to a number of security threats~\cite{alaba2017_IoT_Security_Survey}.

Botnets are one of the major security issues faced in the current IoT landscape~\cite{silva2013_Botnet_Survey}. Botnets are a network of devices corrupted by malware that are ready to be instructed by a bot-master through some Command and Control infrastructure. Once in control of a large enough botnet, botmasters may either use or trade the attacking capabilities in exchange for money to perform targeted attacks on websites~\cite{kolias2017_Mirai_DDoS}. The relatively low computational resources of cheap IoT devices imply the lack of built-in measures to protect against malware~\cite{bertino2017_Botnets_IoT}. Furthermore, IoT devices are often deployed with weak and/or default credentials, and the embedded nature of the devices can make it challenging, and often impossible, to patch vulnerabilities. All this comes together to make IoT devices ideal candidates for botnet infection.

While bots in an IoT botnet do not individually present a threat, a large enough botnet can cause catastrophic impacts. One such example is the Mirai botnet, first identified in 2016, that launched Distributed Denial of Service (DDoS) attacks against security blog KrebsOnSecurity and French cloud computing company OVH, with malicious traffic peaking at 620 Gbps and 1.1 Tbps respectively~\cite{kolias2017_Mirai_DDoS}. More recently, the largest ever DDoS attack was targeted at Amazon Web Services in February 2020, which saw sustained traffic at 2.3 Tbps~\cite{cloudflare_DDoS}. While the majority of the botnet exploits involve DDoS attacks, they are not limited to DDoS attacks only. The following lists several malicious capabilities of botnets:
\begin{enumerate}
    \item DDoS: The most common botnet attack. A large enough botnet can flood network endpoints or links with enough traffic to severely degrade or completely disallow legitimate traffic through the targeted location~\cite{silva2013_Botnet_Survey}.
    \item General bot traffic: Continuous communication and propagation consumes network bandwidth and often results in decreased performance in infected devices.
    \item Spam or Malware dissemination: Rather than sending all bot traffic to one location like in a DDoS attack, botnets can also be used to distribute malicious payloads to a wide variety of targets~\cite{silva2013_Botnet_Survey}.
    \item Firmware corruption: Botnets like BrickerBot can, once commanded, access and destroy a device's firmware~\cite{kolias2017_Mirai_DDoS}.
\end{enumerate}
Research on botnets has surged in popularity after the emergence of the Mirai Botnet, and our research intends to build on top of these efforts.

The rest of this proposal is structured as follows. Section~\ref{sec:problem_statement} explores the motivation behind our study and presents our research objectives. Section~\ref{sec:work_plan} outlines how we plan to approach our proposed modeling and analysis through various stages of the project. Finally, Section~\ref{sec:expected_final_outcome} describes our envisioned expected outcomes from the project.
