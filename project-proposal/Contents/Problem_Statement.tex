The botnet infection process has been studied in great detail and modeled using several modeling formalisms, including epidemiological, stochastic, and economic models~\cite{wainwright2019_Botnet_Models}, as well as Petri Nets~\cite{tanaka2019_PN_Botnet}. As countermeasures, patching vulnerable devices, frequent rebooting, and even the use of innocuous botnets have been suggested to reduce the botnet infection rates~\cite{tanaka2019_PN_Botnet}. However, these studies do not consider different classes of IoT devices that exhibit varying behaviors, and as such, ignore the presence of the heterogeneity inherently present in the massive IoT infrastructure.


In this project, we aim to model a Mirai-like botnet infection process using timed automata and observe the impact of the infection on the simulated network in varying system configurations. The timed automata mathematical model allows us to formally model and analyze the real-time behavior of botnets and its individual components, and observe the behavioral changes in the botnet due to behavioral changes in individual entities within the network~\cite{alur1994_Theory_Timed_Automata}. The high decidability of timed automata allows us to exhaustively check the model for reachability properties such as whether in each simulation scenario, the botnet is capable of infecting all the devices within the entire network, and if not, why. Another consideration would be identifying the percentage of devices that can consistently be infected within a certain time in each system configuration, and how implementing existing security mechanisms impact those figures.



Possible variations in the system configuration include the type and number of such types of devices connected to the network, the capability to reboot and frequency of reboots, vendor patching, and the inclusion of randomness in device behaviors. If feasible, possible extensions to the study may include different network topologies and the modeling of different classes of botnets.