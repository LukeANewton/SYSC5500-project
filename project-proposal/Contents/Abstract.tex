The Internet of Things (IoT) has stimulated the creation of a new era with the promise of ubiquitous connectivity. Recent years have seen a proliferation of cheap IoT devices, many of which employ energy-efficient designs, mandating low computational resources that preclude the use of rigorous security mechanisms. Furthermore, many deployed IoT devices continue to use default vendor passwords, and many others tend to have weak credentials. Adversaries in IoT exploit these security vulnerabilities to create their own army of connected IoT devices, known as botnets, only to be used later to perform various malicious attacks on the network. In this project, we aim to model and simulate the dynamic behavior of a botnet as a networked timed automata using the modeling tool UPPAAL to gain insight into the botnet infection process in diverse system configurations. Additionally, we aim to analyze the impact of the infection process on the modeled network by implementing existing security mechanisms and assess the feasibility of such measures.